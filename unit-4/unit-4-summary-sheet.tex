\documentclass{article}
\usepackage[landscape]{geometry}
\usepackage{url}
\usepackage{multicol}
\usepackage{amsmath}
\usepackage{esint}
\usepackage{amsfonts}
\usepackage{tikz}
\usetikzlibrary{trees}
\usetikzlibrary{decorations.pathmorphing}
\usepackage{amsmath,amssymb}
\usepackage{graphicx}
\usepackage{float}
\usepackage{colortbl}
\usepackage{xcolor}
\usepackage{mathtools}
\usepackage{amsmath,amssymb}
\usepackage{enumitem}
\usepackage[symbol]{footmisc}
\makeatletter

\newcommand*\bigcdot{\mathpalette\bigcdot@{.5}}
\newcommand*\bigcdot@[2]{\mathbin{\vcenter{\hbox{\scalebox{#2}{$\m@th#1\bullet$}}}}}
\renewcommand{\thempfootnote}{\fnsymbol{mpfootnote}}
\makeatother

\title{Unit 1 Summary Sheet}
\usepackage[T1]{fontenc}
\usepackage[utf8]{inputenc}
\usepackage[english]{babel}

\advance\topmargin-.8in
\advance\textheight3in
\advance\textwidth3in
\advance\oddsidemargin-1.5in
\advance\evensidemargin-1.5in
\parindent0pt
\parskip2pt
\newcommand{\hr}{\centerline{\rule{3.5in}{1pt}}}
%\colorbox[HTML]{e4e4e4}{\makebox[\textwidth-2\fboxsep][l]{texto}
\begin{document}

\begin{center}{\huge{\textbf{Unit 4 Summary Sheet}}}\\
\end{center}
\begin{multicols*}{3}

\tikzstyle{mybox} = [draw=black, fill=white, very thick,
    rectangle, rounded corners, inner sep=10pt, inner ysep=10pt]
\tikzstyle{fancytitle} =[fill=black, text=white, font=\bfseries]

%------------ Introduction to Solutions ---------------
\begin{tikzpicture}
\node [mybox] (box){%
    \begin{minipage}{0.3\textwidth}
    \textbf{Terminology}
    \begin{itemize}
        \item Solution - homogeneous mixture of solvent and one or more solutes.
	\item Solvent - substance that has other substances dissolved in it (in greater quantity).
	\item Solute - substance that is dissolved in solution (in lesser quantity).
	\item Aqueous Solution - water is the solvent.
	\item Miscible - substances that are soluble in each other in any proportion.
	\item Immiscible - substances that do not readily dissolve in each other.
	\item Alloy - a solution of two or more metals.
	\item Amalgam - an alloy (solution) of mercury with other metals.
	\item Solubility - amount of solute that dissolves in a given quantity of solvent at a specific temperature.
	\item Saturated Solution - a solution that contains the maximum quantity of solute at a given temperature and pressure.
	\item Unsaturated Solution - a solution in which more solute can dissolve at a given temperature and pressure.
	\item Soluble - solubility > 1g/100mL of solvent.
	\item Insoluble - solubility < 0.1g/100mL of solvent.
	\item Slightly Soluble - between 0.1g and 1g/100mL.
	\item Rate of Dissolving - speed at which a solute dissolves in a solvent.
	\item Electrolyte - solute that conducts current in aqueous solution.
	\item Non-electrolyte - solute that does not conduct current in aqueous solution.
    \end{itemize}
    \end{minipage}
};
%------------ Introduction to Solutions Header ---------------------
\node[fancytitle, right=10pt] at (box.north west) {Introduction to Solutions};
\end{tikzpicture}

%------------ Rate of Dissolving ---------------
\begin{tikzpicture}
\node [mybox] (box){%
    \begin{minipage}{0.3\textwidth}
    \textbf{Factors Affecting Rate of Dissolving}:
    \begin{enumerate}
        \item Temperature: $\uparrow$ average kinetic energy = $\uparrow$ number of collisions
        \item Agitation: $\uparrow$ contact between solute and solvent
	\item Particle Size: $\downarrow$ particle size = $\uparrow$ surface area per unit volume (or mass)
    \end{enumerate}
    \end{minipage}
};
%------------ Rate of Dissolving Header ---------------------
\node[fancytitle, right=10pt] at (box.north west) {Rate of Dissolving};
\end{tikzpicture}

%------------ Factors of Solubility ---------------
\begin{tikzpicture}
\node [mybox] (box){%
    \begin{minipage}{0.3\textwidth}
    \textbf{Factors Affecting Solubility}:
    \begin{enumerate}
        \item Molecule Size: $\uparrow$ molecule size = $\downarrow$ solubility
        \item Temperature:
	\begin{itemize}
	    \item Solid: $\uparrow$ absolute temperature causes bonds to become weaker = $\uparrow$ solubility
	    \item Gas: $\uparrow$ absolute temperature causes intermolecular bonds to break = $\downarrow$ solubility
	\end{itemize}
	\item Pressure: $\uparrow$ pressure on gas = $\uparrow$ solubility
    \end{enumerate}
    \end{minipage}
};
%------------ Factors of Solubility Header ---------------------
\node[fancytitle, right=10pt] at (box.north west) {Factors of Solubility};
\end{tikzpicture}

%------------ Dissolving Process ---------------
\begin{tikzpicture}
\node [mybox] (box){%
    \begin{minipage}{0.3\textwidth}
    \begin{enumerate}
        \item Solute particles separate (heat is absorbed).
        \item Solvent particles separate by breaking intermolecular forces (heat is absorbed).
	\item Attraction between solute and solvent particles occur (heat is released).
    \end{enumerate}
    e.g. sodium chloride (salt) dissociating into sodium and chloride ions when it is mixed with water. \\ \\
    \textbf{Hydrogen bonding} accounts for many of the unique physical properties of water.
    \begin{itemize}
        \item high melting and boiling points
        \item high surface tension
	\item ability to exchange thermal energy with little change in temperature
        \item inability to mix with non-polar compounds
    \end{itemize}
    \end{minipage}
};
%------------ Dissolving Process Header ---------------------
\node[fancytitle, right=10pt] at (box.north west) {Dissolving Process};
\end{tikzpicture}

%------------ Solutions and Their Characteristics ---------------
\begin{tikzpicture}
\node [mybox] (box){
    \begin{minipage}{0.3\textwidth}
    \small{
    \hspace*{-0.5cm}
    \begin{tikzpicture}[level distance=2cm,
        level 1/.style={sibling distance=4.1cm},
	level 2/.style={sibling distance=2.1cm}]
	\centering
	\node {Matter}
	    child {node {Mixture}
	    child {node {\begin{tabular}{c} Homogeneous \\ Mixture \\ (solution) \end{tabular}}}
	    child {node {\begin{tabular}{c} Heterogeneous \\ Mixture \end{tabular}}
	    child {node {Colloids}}
	    child {node {Suspensions}}
	        }
	    }
	    child {node {Pure Substance}
	    child {node {Compound}}
	    child {node {Element}}
	    };
    \end{tikzpicture}}
    \textbf{Homogeneous Mixture}: a mixture in which the composition is uniform throughout the mixture. \\
    e.g. wine, coffee, apple juice, salt water. \\
    \textbf{Heterogeneous Mixture}: a mixture that contains two or more phases. \\
    e.g. blood, paint, sand, oil and water.
    \begin{itemize}
        \item Colloid $\rightarrow$ has a homogeneous appearance but is not transparent (e.g. milk, butter, etc.)
	\item Suspension $\rightarrow$ particles are visible to the naked eye; they can be separated by gravity (e.g. flour in water, muddy water, etc.)
    \end{itemize}
    \textbf{Variable Composition}: different ratios of solvent to solute are possible. \\
    \textbf{Concentration}: the ratio of the quantity of solute to the quantity of solution or solvent. \\
    \textbf{Concentrated Solution}: a solution with a relatively large quantity of solute dissolved per unit volume of solution. \\
    \textbf{Dilute Solution}: a solution with a relatively small quantity of solute dissolved per unit volume of solution.
    \end{minipage}
};
%------------ Solutions and Their Characteristics Header ---------------------
\node[fancytitle, right=10pt] at (box.north west) {Solutions and Their Characteristics};
\end{tikzpicture}

%------------ Degrees of Saturation ---------------
\begin{tikzpicture}
\node [mybox] (box){%
    \begin{minipage}{0.3\textwidth}
    \textbf{Supersaturated Solution}: a solution that contains more than the maximum quantity of solute that it should at a given temperature and pressure. \\
    This can be attained by dissolving excess solute at a \textit{higher temperature}, and then \textit{cooling} the solution.
    \end{minipage}
};
%------------ Degrees of Saturation Header ---------------------
\node[fancytitle, right=10pt] at (box.north west) {Degrees of Saturation};
\end{tikzpicture}

%------------ Process of Dissolving ---------------
\begin{tikzpicture}
\node [mybox] (box){%
    \begin{minipage}{0.3\textwidth}
    \textbf{Process of Dissolving}: the ability of a solute to be dissolved in a solvent depends on the forces of attraction that are present between the particles. \\
    The \textbf{3 intermolecular forces} present:
    \begin{itemize}
        \item attraction between 2 solute particles
        \item attraction between 2 solvent particles
        \item attraction between a solute and a solvent particle
    \end{itemize}
    Process of Dissolving at the Molecular Level:
    \begin{enumerate}
        \item Forces between the particles in the solid (solute) must be broken.
	\begin{itemize}
	    \item For ionic solid, the ion-ion forces which are holding the ions together must be broken.
	    \item For molecular solid, the London dispersion, dipole-dipole, or hydrogen bonds must be broken.
	\end{itemize}
	\item Some intermolecular forces between the liquid (solvent) particles must be broken.
	\item Attractions form between solid (solute) particles and liquid (solvent) particles.
    \end{enumerate}
    \textit{The intermolecular forces between solute and solvent must be great enough to overcome the attractions of solute-solute and solvent-solvent, in order for a solute to dissolve in a solvent}. \\ \\
    \textbf{Solubility Rule}: polar compounds dissolve in polar solvents, and non-polar compounds dissolve in non-polar solvents (like dissolves like). \\
    \textbf{Ion-dipole Interactions}
    \begin{itemize}
        \item attractive force between an ion and a polar molecule
	\item typically an ionic compound will dissolve in a polar solvent (exceptions occur with compounds that have very strong ionic bonds)
    \end{itemize}
    Dissociation of sodium chloride ($NaCl$) in water: \\
    $NaCl(s)\rightarrow Na^{+}(aq)+Cl^{-}(aq)$ \\
    \textit{Water does not undergo a chemical change, which is why it is not included}.
    \end{minipage}
};
%------------ Process of Dissolving Header ---------------------
\node[fancytitle, right=10pt] at (box.north west) {Process of Dissolving};
\end{tikzpicture}

%------------ Solubility of Ionic Compounds in Water ---------------
\begin{tikzpicture}
\node [mybox] (box){%
    \begin{minipage}{0.3\textwidth}
    \textbf{Dissociation}: the separation of individual ions from an ionic compound as it dissolves in water. \\
    \textbf{Hydrated Ions}: ions that are surrounded by water molecules in an aqueous solution.
    \begin{itemize}
        \item the ability of hydrated ions to move is what accounts for the electrical conductivity of a solution (electrolytes)
	\item since water is polar, most non-polar covalent compounds are not soluble in water
	\item molecules do not separate into individuals atoms, they are only surrounded by water
	\item these molecules are neutral, so they do not conduct electricity (non-electrolytes)
    \end{itemize}
    \end{minipage}
};
%------------ Solubility of Ionic Compounds in Water Header ---------------------
\node[fancytitle, right=10pt] at (box.north west) {Solubility of Ionic Compounds in Water};
\end{tikzpicture}

%------------ Surfactant ---------------
\begin{tikzpicture}
\node [mybox] (box){%
    \begin{minipage}{0.3\textwidth}
    \textbf{Hydrophobic}: the end that is ``water-fearing'', which is not attracted to water. \\
    e.g. non-polar substances such as oil, grease, etc. \\
    \textbf{Hydrophilic}: the end that is ``water-loving'', which is attracted to water. \\
    e.g. polar substances such as water, etc. \\
    \textbf{Surfactant}: a substance that can reduce the surface tension of a solvent; it has a hydrophobic part and a hydrophilic part. \\
    e.g. soaps, detergents, lubricants, fabric softeners, inks, adhesives, etc.
    \end{minipage}
};
%------------ Surfactant Header ---------------------
\node[fancytitle, right=10pt] at (box.north west) {Surfactant};
\end{tikzpicture}

%------------ Solubility Curves ---------------
\begin{tikzpicture}
\node [mybox] (box){%
    \begin{minipage}{0.3\textwidth}
    \textbf{Solubility Curve}: a graph of the solubility of a substance over a range of temperatures. \\
    220 g of sucrose dissolves in 100 g of water at $30^{\circ}C$.
    \begin{itemize}
	\item the curve represents the maximum amount of solute that will dissolve in 100 g of water at that temperature
        \item ionic compounds - solubility of most ionic compounds increases with rising temperature
        \item gases - solubility of gases decreases as the temperature rises
    \end{itemize}
    \textbf{Thermal Pollution}: is excess thermal energy released into water; can reduce oxygen concentrations.
    \end{minipage}
};
%------------ Solubility Curves Header ---------------------
\node[fancytitle, right=10pt] at (box.north west) {Solubility Curves};
\end{tikzpicture}

%------------ Concentration of Solutions ---------------
\begin{tikzpicture}
\node [mybox] (box){%
    \begin{minipage}{0.3\textwidth}
    \textbf{Concentration}: the ratio of the amount of solute per quantity of solution.
    $$concentration=\frac{quantity\,of\,solute}{quantity\,of\,solution}$$
    \textbf{Percentage Volume/Volume (\% v/v)} \\
    Generally used when mixing two liquids to form a solution.
    $$\%v/v=\frac{v_{solute}}{v_{solution}}\times100\%$$
    \textbf{Percentage Mass/Volume (\% m/v)} \\
    Generally used when mixing a solid solute into a liquid to form a solution. \\
    \textit{Units must be kg and L, or g and mL}.
    $$\%m/v=\frac{m_{solute}}{v_{solution}}\times100\%$$
    \textbf{Percentage Mass/Mass (\% m/m)} \\
    The amount of grams of solute per 100 g of solution is numerically the same as the mass/mass percentage.
    $$\%m/m=\frac{m_{solute}}{m_{solution}}\times100\%$$
    \textbf{Parts Per Million (PPM)} \\
    1 ppm = 1 mg/L, is used for dilute solutions. \\
    Density of water = 1.0 g/mL.
    $$ppm=\frac{m_{solute}}{m_{solution}}\times10^{6}$$
    \textbf{Parts Per Billion (PPB)} \\
    1000 ppb = 1 mg/L and 1 ppm = 1000 ppb.
    $$ppb=\frac{m_{solute}}{m_{solution}}\times10^{9}$$
    \end{minipage}
};
%------------ Concentration of Solutions Header ---------------------
\node[fancytitle, right=10pt] at (box.north west) {Concentration of Solutions};
\end{tikzpicture}

%------------ Molar Concentration ---------------
\begin{tikzpicture}
\node [mybox] (box){%
    \begin{minipage}{0.3\textwidth}
    \textbf{Molar Concentration (Molarity)}: the number of moles of solute dissolved in 1 L of solution.
    $$molar\,concentration=\frac{amount\,of\,solute\,(mol)}{amount\,of\,solution\,(L)}$$
    Can also be written as $\displaystyle{c=\frac{n}{V}}$, units are mol/L or M. \\
    In stoichiometry problems, use $n=cV$ to calculate the number of moles, and M as a conversion factor.
    \end{minipage}
};
%------------ Molar Concentration Header ---------------------
\node[fancytitle, right=10pt] at (box.north west) {Molar Concentration};
\end{tikzpicture}

\end{multicols*}
\end{document}
