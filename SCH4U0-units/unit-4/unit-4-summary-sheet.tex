\documentclass{article}
\usepackage[landscape]{geometry}
\usepackage{url}
\usepackage{multicol}
\usepackage{amsmath}
\usepackage{esint}
\usepackage{amsfonts}
\usepackage{tikz}
\usetikzlibrary{decorations.pathmorphing}
\usepackage{amssymb}
\usepackage[version=4]{mhchem}
\usepackage{graphicx}
\usepackage{float}
\usepackage{colortbl}
\usepackage{xcolor}
\usepackage{mathtools}
\usepackage{enumitem}
\usepackage[symbol]{footmisc}
\usepackage{booktabs}
\makeatletter

\newcommand*\bigcdot{\mathpalette\bigcdot@{.5}}
\newcommand*\bigcdot@[2]{\mathbin{\vcenter{\hbox{\scalebox{#2}{$\m@th#1\bullet$}}}}}
\renewcommand{\thempfootnote}{\fnsymbol{mpfootnote}}
\makeatother

\title{Unit 4 - Chemical Equilibrium}
\usepackage[T1]{fontenc}
\usepackage[utf8]{inputenc}
\usepackage[english]{babel}

\advance\topmargin-.8in
\advance\textheight3in
\advance\textwidth3in
\advance\oddsidemargin-1.5in
\advance\evensidemargin-1.5in
\parindent0pt
\parskip2pt
\newcommand{\hr}{\centerline{\rule{3.5in}{1pt}}}
%\colorbox[HTML]{e4e4e4}{\makebox[\textwidth-2\fboxsep][l]{texto}
\begin{document}

\begin{center}{\huge{\textbf{Unit 4 - Chemical Equilibrium}}}\\
\end{center}
\begin{multicols*}{3}

\tikzstyle{mybox} = [draw=black, fill=white, very thick,
    rectangle, rounded corners, inner sep=10pt, inner ysep=10pt]
\tikzstyle{fancytitle} =[fill=black, text=white, font=\bfseries]

%------------ Dynamic Equilibrium ---------------
\begin{tikzpicture}
\node [mybox] (box){%
    \begin{minipage}{0.3\textwidth}
    \textbf{Dynamic Equilibrium}: a chemical system at equilibrium that is changing at the molecular level, while its macroscopic properties remain constant. \\
    \textbf{Chemical Equilibrium}: a state in a chemical system in which the forward reaction and the reverse reaction are occurring at the same rate. \\
    \textbf{Reversible Reaction}: a chemical reaction that proceeds in both the forward and reverse directions. \\
    \textbf{Homogeneous Equilibrium}: a chemical system in equilibrium in which all of the components are in the same physical state. \\
    e.g. \ce{N2(g) + 3H2(g) <=> 2NH3(g)} \\
    \textbf{Heterogeneous Equilibrium}: a chemical system in equilibrium in which the components are in different physical states. \\
    e.g. \ce{CaCO3(s) <=> CaO(s) + CO2(g)}
    \end{minipage}
};
%------------ Dynamic Equilibrium Header ---------------------
\node[fancytitle, right=10pt] at (box.north west) {Dynamic Equilibrium};
\end{tikzpicture}

%------------ Le Châtelier's Principle ---------------
\begin{tikzpicture}
\node [mybox] (box){%
    \begin{minipage}{0.3\textwidth}
    \textbf{Le Châtelier's Principle}: a principle stating that if a chemical system in a state of equilibrium is disturbed, the system will undergo a change that shifts its equilibrium position in a direction that reduces the effect of the disturbance. \\
    The effects of concentration changes:
    \begin{itemize}
        \item Increasing the concentration of a product causes a shift to reactant formation.
        \item Decreasing the concentration of a product causes a shift to product formation.
        \item Increasing the concentration of a reactant causes a shift to product formation.
        \item Decreasing the concentration of a reactant causes a shift to reactant formation.
	\item $K\textsubscript{eq}$ is not affected by changes in the product or reaction concentrations, because the ratio of products to reactants is the same.
    \end{itemize}
    e.g. your body controls the concentration of carbonic acid in your blood by removing carbon dioxide when you exhale, causing a shift in the reaction rates. \\
    \ce{H2CO3(aq) <=> CO2(aq) + H2O(l)}
    \end{minipage}
};
%------------ Le Châtelier's Principle Header ---------------------
\node[fancytitle, right=10pt] at (box.north west) {Le Châtelier's Principle};
\end{tikzpicture}

%------------ Le Châtelier's Principle Continued ---------------
\begin{tikzpicture}
\node [mybox] (box){%
    \begin{minipage}{0.3\textwidth}
    The effects of temperature changes:
    \begin{itemize}
        \item A temperature increase (addition of thermal energy) favours the endothermic (heat-absorbing) reaction, whereas a temperature decrease (removal of thermal energy) favours the exothermic (heat-releasing) reaction.
        \item In an endothermic system, an increase in temperature increases $K\textsubscript{eq}$.
        \item In an exothermic system, an increase in temperature decreases $K\textsubscript{eq}$.
    \end{itemize}
    e.g. the decomposition of \ce{N2O4(g)}, is endothermic. If this chemical system is heated, the energy in the system increases. There is more energy available for the forward reaction to occur. \\
    \ce{$\underset{\text{colourless}}{\ce{N2O4(g) + 58.0 kJ/mol}}$ <=> $\underset{\text{brown}}{\ce{2NO2(g)}}$} \\
    The effects of pressure or volume changes:
    \begin{itemize}
        \item If pressure increases (and therefore volume decreases), the reaction shifts so that the total number of particles decreases, which decreases the pressure in the system.
        \item If pressure decreases (and therefore volume increases), the reaction shifts so that the total number of particles increases, which increases the pressure in the system.
        \item An inert gas added to a chemical system at equilibrium does not cause a shift in either direction in a reversible reaction.
	\item $K\textsubscript{eq}$ is not affected by a change in pressure or volume because the ratio of products to reactants is the same.
    \end{itemize}
    e.g. instant coffee is an example of food that can be preserved by freeze-drying (removing moisture helps preserve it), a process that uses a shift in equilibrium. \\
    \ce{H2O(s) <=> H2O(g)} \\
    $\star$ Adding a catalyst causes a system to reach equilibrium more quickly, but it does not shift the equilibrium in either direction.
    \end{minipage}
};
%------------ Le Châtelier's Principle Continued Header ---------------------
\node[fancytitle, right=10pt] at (box.north west) {Le Châtelier's Principle Continued};
\end{tikzpicture}

%------------ Equilibrium Law ---------------
\begin{tikzpicture}
\node [mybox] (box){%
    \begin{minipage}{0.3\textwidth}
    \textbf{Law of Chemical Equilibrium} or \textbf{Law of Mass Action}: in a chemical system at equilibrium, there is a constant ratio between the concentrations of the products and the concentrations of the reactants. \\
    \textbf{The Equilibrium Constant Expression} \\
    For a general equilibrium reaction: \\
    \ce{aA + bB <=> cC + dD} \\ \\
    $\displaystyle{K\textsubscript{eq}=\frac{[C]^{c}[D]^{d}}{[A]^{a}[B]^{b}}}$ or $\displaystyle{K\textsubscript{eq}=\frac{[\text{products}]}{[\text{reactants}]}}$
    \begin{itemize}
        \item {[A]}, {[B]}, {[C]}, and {[D]} represent the concentrations of the reactants and products after the reaction has reached equilibrium and the concentrations are no longer changing.
	\item The exponents a, b, c, and d, are the stoichiometric coefficients from the balanced chemical equation at equilibrium.
    \end{itemize}
    \end{minipage}
};
%------------ Equilibrium Law Header ---------------------
\node[fancytitle, right=10pt] at (box.north west) {Equilibrium Law};
\end{tikzpicture}

%------------ Equilibrium Constant ---------------
\begin{tikzpicture}
\node [mybox] (box){%
    \begin{minipage}{0.3\textwidth}
    \textbf{Equilibrium Constant, $K\textsubscript{eq}$}: the ratio of equilibrium concentrations for a particular chemical system at a particular temperature. \\
     $\rightarrow$ A \textbf{reaction quotient}, $Q\textsubscript{eq}$ or $Q\textsubscript{p}$, has the same formula as $K\textsubscript{eq}$ or $K\textsubscript{p}$, but the chemical system may or may not be at equilibrium. \\
    Three possible situations that occur when using the reaction quotient:
    \begin{itemize}
        \item If $Q\textsubscript{eq} < K\textsubscript{eq}$ (or $Q\textsubscript{p} < K\textsubscript{p}$), the ratio of products to reactants is less than $K\textsubscript{eq}$. To reach equilibrium, more products must form and reactants must be consumed. The reaction shifts to the right (toward product formation) to reach equilibrium.
	\item If $Q\textsubscript{eq} = K\textsubscript{eq}$ (or $Q\textsubscript{p} = K\textsubscript{p}$), the system is at equilibrium.
	\item If $Q\textsubscript{eq} > K\textsubscript{eq}$ (or $Q\textsubscript{p} > K\textsubscript{p}$), the ratio of products to reactants is greater than $K\textsubscript{eq}$. Products must decompose into reactants to reach equilibrium. The reaction shifts to the left (toward reactant formation) to reach equilibrium.
    \end{itemize}
    \end{minipage}
};
%------------ Equilibrium Constant Header ---------------------
\node[fancytitle, right=10pt] at (box.north west) {Equilibrium Constant};
\end{tikzpicture}

%------------ Solubility Product ---------------
\begin{tikzpicture}
\node [mybox] (box){%
    \begin{minipage}{0.3\textwidth}
    \textbf{Common-Ion Effect}: the shift in equilibrium position caused by the addition of a compound that has an ion in common with one of the dissolved substances. \\
    \textbf{Solubility-Product Constant, $K\textsubscript{sp}$}: an equilibrium constant for slightly soluble ionic compounds.
    \begin{itemize}
        \item If $Q\textsubscript{sp} < K\textsubscript{sp}$, the solution is unsaturated and no precipitate forms.
        \item If $Q\textsubscript{sp} = K\textsubscript{sp}$, the solution is saturated and no change occurs.
        \item If $Q\textsubscript{sp} > K\textsubscript{sp}$, a precipitate forms until the solution is saturated.
    \end{itemize}
    \end{minipage}
};
%------------ Solubility Product Header ---------------------
\node[fancytitle, right=10pt] at (box.north west) {Solubility Product};
\end{tikzpicture}

%------------ Acid-Base Equilibrium ---------------
\begin{tikzpicture}
\node [mybox] (box){%
    \begin{minipage}{0.3\textwidth}
    \textbf{Dissociation}: the process of breaking apart into smaller particles, such as ions or neutral particles. \\
    \textbf{Hydronium Ion, \ce{H3O+}}: a proton bonded to a water molecule by a covalent bond. \\
    \textbf{Hydroxide Ion, \ce{OH-}}: a negatively charged ion composed of a hydrogen atom and an oxygen atom. \\
    \textbf{Limitations of the Arrhenius Theory}:
    \begin{itemize}
        \item Some substances do not have \ce{OH} in their chemical formulas, but nevertheless they yield hydroxide ions when they react with water.
	\item Only applies to aqueous solutions.
	\item States that hydrogen ions will dissociate and float freely in water.
	\item Different acids create different pH, even when they have the same concentration.
    \end{itemize}
    e.g. \ce{NH3(aq) + H2O(l) <=> NH4^{+1}(aq) + OH^{-1}(aq)} \\
    \textbf{Brønsted-Lowry Theory of Acids and Bases}:
    \begin{itemize}
        \item A Brønsted-Lowry acid is a \textit{proton donor} or any substance that donates a hydrogen ion. An acid must contain hydrogen in its formula. All Arrhenius acids are also Brønsted-Lowry acids.
	\item A Brønsted-Lowry base is a \textit{proton acceptor} or any substance that accepts a hydrogen ion. A base must have a lone pair of electrons to bind with the hydrogen ion. All Arrhenius bases are also Brønsted-Lowry bases.
    \end{itemize}
    \end{minipage}
};
%------------ Acid-Base Equilibrium Header ---------------------
\node[fancytitle, right=10pt] at (box.north west) {Acid-Base Equilibrium};
\end{tikzpicture}

%------------ Acid-Base Equilibrium Continued ---------------
\begin{tikzpicture}
\node [mybox] (box){%
    \begin{minipage}{0.3\textwidth}
    $\rightarrow$ According to the Brønsted-Lowry definition, acids and bases do not need to be in aqueous solution and their activity in water is not part of the definition. \\
    $\rightarrow$ In general, an acid-base equilibrium reaction shifts in the direction in which a stronger acid and a stronger base form a weaker acid and a weaker base. \\
    $\rightarrow$ Any substance that behaves as an acid can do so only if another substance behaves as a base at the same time. \\
    \textbf{Conjugate Acid-Base Pair}: two substances that are related by the gain or loss of a proton; the acid of an acid-base pair has one more proton than its conjugate base. \\
    \textbf{Conjugate Base}: the particle that is produced when an acid donates a hydrogen ion to a base. \\
    \textbf{Conjugate Acid}: the particle that is produced when a base accepts a hydrogen ion from an acid. \\
    \textbf{Amphoteric Substance}: a molecule or ion that can act as an acid or base. \\
    e.g. water can act as an acid or as a base. \\
    \ce{$\underset{\text{acid}}{\ce{HCl(aq)}}$ + $\underset{\text{base}}{\ce{H2O(l)}}$ -> $\underset{\text{conjugate base}}{\ce{Cl-(aq)}}$ + $\underset{\text{conjugate acid}}{\ce{H3O+(aq)}}$} \\
    \ce{$\underset{\text{base}}{\ce{NH3(aq)}}$ + $\underset{\text{acid}}{\ce{H2O(l)}}$ -> $\underset{\text{conjugate acid}}{\ce{NH4+(aq)}}$ + $\underset{\text{conjugate base}}{\ce{OH-(aq)}}$} \\
    \textbf{Ion-Product Constant of Water, $K\textsubscript{w}$}: the equilibrium constant for the autoionization of water. \\
    $\star$ For a reaction that is the sum of two or more reactions, the overall equilibrium constant is the product of the individual equilibrium constants. $K\textsubscript{a}K\textsubscript{b}=K\textsubscript{w}$
    \end{minipage}
};
%------------ Acid-Base Equilibrium Continued Header ---------------------
\node[fancytitle, right=10pt] at (box.north west) {Acid-Base Equilibrium Continued};
\end{tikzpicture}

%------------ pH and pOH ---------------
\begin{tikzpicture}
\node [mybox] (box){%
    \begin{minipage}{0.3\textwidth}
    \textbf{pH}: the negative common logarithm of the concentration of the hydronium ion, \ce{H3O+}. \\
    $\text{pH}=-\log\ce{[H3O+]}$ and $\ce{[H3O+]}=10^{-\text{pH}}$ \\
    $\rightarrow$ Sometimes, \ce{H3O+(aq)} is simplified to \ce{H+(aq)}. \\
    \textbf{pOH}: the negative common logarithm of the concentration of the hydroxide ion, \ce{OH-}. \\
    $\text{pOH}=-\log\ce{[OH^-]}$ and $\ce{[OH^-]}=10^{-\text{pOH}}$ \\
    $\rightarrow$ A high pH number corresponds to a low concentration of the hydronium ion or a basic solution. \\
    $\rightarrow$ A low pH number corresponds to a high concentration of the hydronium ion or an acidic solution. \\
    $K\textsubscript{w}=\ce{[H3O+]}\ce{[OH^-]}=1.0\times10^{-14}$ (at 25$^{\circ}$C) \\
    $\text{p}K\textsubscript{w}=\text{pH}+\text{pOH}=14.00$ (at 25$^{\circ}$C) \\
    $\therefore$ The sum of pH and pOH is 14.00 for any aqueous solution at 25$^{\circ}$C.
    \end{minipage}
};
%------------ pH and pOH Header ---------------------
\node[fancytitle, right=10pt] at (box.north west) {pH and pOH};
\end{tikzpicture}

%------------ Solving Equilibrium Concentrations ---------------
\begin{tikzpicture}
\node [mybox] (box){%
    \begin{minipage}{0.3\textwidth}
    \textbf{I}nitial, \textbf{C}hange, \textbf{E}quilibrium (ICE) table, e.g.
    \begin{center}
    \begin{tabular}{cc@{}c@{}c@{}c@{}c}
    \toprule
    & \ce{[H2(g)]} & ${}+{}$ & \ce{[I2(g)]} & \ce{<=>} & \ce{[HI(g)]} \\
    \midrule
    I & 2.00 && 1.00 && 0 \\
    C & $-x$ && $-x$ && $+2x$ \\
    E & $2.00-x$ && $1.00-x$ && $2x$ \\
    \bottomrule
    \end{tabular}
    \end{center}
    \underline{Case 1} \\
    Use the $K$ value, reaction quotient and equilibrium law, an ICE chart, and take square root of both sides. \\
    \underline{Case 2} \\
    Check assumption by calculating if $\displaystyle{\frac{[C]_{i}}{K}>>100}$. \\
    Check if $\displaystyle{\frac{\Delta}{[C]_{i}}}$ is less than 5\% to verify assumption. \\
    \underline{Case 3} \\
    If the equation cannot be rooted and the assumption is not valid, then the quadratic formula must be used.
    \end{minipage}
};
%------------ Solving Equilibrium Concentrations Header ---------------------
\node[fancytitle, right=10pt] at (box.north west) {Solving Equilibrium Concentrations};
\end{tikzpicture}

%------------ Acid-Base Strength ---------------
\begin{tikzpicture}
\node [mybox] (box){%
    \begin{minipage}{0.3\textwidth}
    \textbf{Strong Acid}: an acid that dissociates completely into ions in water. The 6 strong acids are: \ce{HCl(aq)}, \ce{HBr(aq)}, \ce{HI(aq)}, \ce{HClO4(aq)}, \ce{HNO3(aq)}, \ce{H2SO4(aq)} \\
    \textbf{Weak Acid}: an acid that ionizes to a limited extent in water. e.g. \ce{NH4+(aq)}, \ce{H2S(aq)}, \ce{H3PO4(aq)} \\
    \textbf{Strong Base}: A base that ionizes completely in water. Some common strong bases are: \ce{LiOH(aq)}, \ce{NaOH(aq)}, \ce{KOH(aq)}, \ce{Ca(OH)2(aq)}, \ce{Ba(OH)2(aq)} \\
    \textbf{Weak Base}: A base that ionizes to a limited extent in water. e.g. \ce{NH3(aq)}, \ce{PO4^{3-}(aq)}, \ce{OCl-(aq)} \\
    $\rightarrow$ An acid or base is considered strong if $K\textsubscript{a} > 1$. \\
    $\rightarrow$ An acid or base is considered weak if $K\textsubscript{a} < 1$. \\
    \textbf{Acid-Dissociation Constant} or \textbf{Acid-Ionization Constant, $K\textsubscript{a}$}: an equilibrium constant for the dissociation of an acid into ions or for the ionization of an acid.
    \begin{center}\ce{stronger acid -> higher [H3O+] -> larger $K\textsubscript{a}$ value}\end{center}
    \textbf{Base-Dissociation Constant} or \textbf{Base-Ionization Constant, $K\textsubscript{b}$}: an equilibrium constant for the ionization of a base. \\
    Consider the reaction of acetic acid, \ce{CH3COOH(aq)}: \\ \\
    $\displaystyle{K\textsubscript{a}=\frac{\ce{[H3O^+]}\ce{[CH3COO^-]}}{\ce{[CH3COOH]}}}$ $\displaystyle{K\textsubscript{b}=\frac{\ce{[CH3COOH]}\ce{[OH^-]}}{\ce{[CH3COO^-]}}}$
    \end{minipage}
};
%------------ Acid-Base Strength Header ---------------------
\node[fancytitle, right=10pt] at (box.north west) {Acid-Base Strength};
\end{tikzpicture}

\end{multicols*}
\end{document}
